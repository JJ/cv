%!TEX TS-program = xelatex
\documentclass[]{friggeri-jj-cv}
\addbibresource{bibliography.bib}
\begin{document}
\header{JJ}{Merelo}
       {Professor at University of Granada}

% In the aside, each new line forces a line break
\begin{aside}
  \section{about}
    Depto. Arquitectura y Tecnología de Computadores
    C/ Daniel Saucedo Aranda s/n
    18071 Granada
    Spain
    ~
    \href{mailto:jjmerelo@gmail.com}{jjmerelo@gmail.com}
    \href{http://twitter.com/jjmerelo}{{\tt @jjmerelo}}
    \href{http://scholar.google.com/citations?user=gFxqc64AAAAJ}{Google Scholar}
    \href{https://www.researchgate.net/profile/JJ_Merelo}{ResearchGate}
    \href{https://figshare.com/authors/Juan_J_Merelo/541327}{FigShare}
   \href{https://github.com/JJ}{GitHub}
   \href{http://lnkd.in/dBVqYPa}{LinkedIn}
   \href{http://rpubs.com/jjmerelo/}{RPubs}
   \href{http://gitxiv.com/users/jj-merelo}{GitXiv}
   \href{https://www.npmjs.com/~jjmerelo}{NPM}
   \href{http://search.cpan.org/~jmerelo/}{CPAN}
  \section{languages}
    Bilingual Spanish/English
    spoken French \& Italian
    German, Dutch notions
  \section{programming}
    {\color{red} \large $\varheartsuit$} Perl
    {\color{red} $\varheartsuit$} JavaScript (including {\tt
      node.js}), R, Perl 6,
    Lua, Python, Ruby, shell, CSS3 \& HTML5, chef, Scala, Ansible,
     Go
\end{aside}

\section{interests}

evolutionary algorithms, complex networks, social networks, emerging
computing, literary engineering, distributed computing, stealth
computing, computational intelligence in games, machine learning,
repository mining

\section{education}

\begin{entrylist}
  \entry
    {1994}
    {Ph.D. {\normalfont in Physics} magna cum laude}
    {University of Granada}
    {\emph{Evolutionary neural networks}}
  \entry
    {1988}
    {M.Sc. {\normalfont in Physics} magna cum laude}
    {University of Granada}
    {Majoring in Physics, Specialization in Theoretical Physics}
  \entry
    {1983–1988}
    {Avg B+}
    {University of Granada}
    {Physics}
\end{entrylist}

\section{experience}

\begin{entrylist}
  \entry
    {1988-2008}
    {Assistant professor}
    {University of Granada}
    {\emph{Dept. of Computer Architecture}}
  \entry
    {2009-}
    {Full professor.}
    {University of Granada}
    {\emph{Dept. of Computer Architecture}}
    \entry
    {2008-}
    {Director.}
    {University of Granada}
    {\href{http://osl.ugr.es}{\emph{Free Software Office}}}
\end{entrylist}

\section{books}
\begin{entrylist}
  \entry
    {2013}
    {Manuel the Magnificent Mechanical Man}
    {\href{http://www.amazon.com/dp/B00ED084BK/}{Kindle Direct Publishing}}
    {\href{http://jj.github.io/hoborg}{An open source novel} and a
      Perl module}
  \end{entrylist}
  
  \begin{entrylist}
  \entry
    {2016}
    {Granada On}
    {\href{https://figshare.com/authors/Juan_J_Merelo/541327}{Kindle Direct Publishing}}
    {\href{http://github.com/granada-off}{An open source travel book}}
  \end{entrylist}

  
\section{awards}
\begin{entrylist}
 \entry
    {2009}
    {Bubok literary prize}
    {\href{http://cultura.elpais.com/cultura/2009/05/06/actualidad/1241560804_850215.html}{1st
        edition of this prize}}
    {Awarded to the novel {\tt lujoyglamour.net}.}
 \entry
    {April 2014}
    {CS4HS award}
    {\href{http://cs4hs.com}{Computer Science 4 High Schools}}
    {developed as the \href{http://cs4hs.ugr.es}{Google/UGR Tech Campus for Girls}}
\end{entrylist}


\section{applications}

\begin{entrylist}
  \entry
    {2004-}
    {{\tt Algorithm::Evolutionary}}
    {\href{http://search.cpan.org/dist/Algorithm-Evolutionary/}{http://search.cpan.org/dist/Algorithm-Evolutionary/}}
    {Evolutionary Algorithm library in
      perl. Check \href{http://search.cpan.org/~jmerelo/}{All my modules at CPAN}}
  \entry
    {2014-}
    {{\tt NodEO}}
    {\href{https://npmjs.org/package/nodeo}{https://www.npmjs.org/package/nodeo}}
    {Evolutionary Algorithm Library in JavaScript for NodeJS}
\end{entrylist}

\section{international projects}

\begin{entrylist}
  \entry
    {2001-2005}
    {{\sf DREAM} Distributed Resource Evolutionary Algorithm Machine}
    {\href{http://www.soc.napier.ac.uk/~benp/dream/dream.htm}{Dream
        Project page}}
    {Peer to peer, asynchronous, evolutionary computation}
  \entry
    {2012-15}
    {Muses project}
    {\href{https://musesproject.eu/}{Muses Project Page}}
    {Mobile, user centered, adaptive security}
\end{entrylist}

\section{publications}

I published my first paper before graduation: 

\cite{merelo88} 
and my first paper in a journal a few years
afterwards, getting already into parallel computing:

\cite{parallel90}
I was working for a few years applying Kohonen's Self-Organizing Map to protein structure prediction

\cite{jjproteng} 
and it hit pay dirt, since 20 years after it gathers more than a
thousand references (and growing). Let's not dwell in the past
and talk about current developments. I was PhD advisor for JLJ Laredo,
who continued with DREAM development. This paper is a good example of
our work in the area.

\cite{evag:gpem}
but his work raised several issues about how asynchronous distributed
evolutionaory computing worked, which we addressed in this paper in
which I am the first author.

\cite{jj:2008:PPSN}
problems similar to what we found in our work on browser-based
evolutionary algorithms.

\cite{agajaj}
long before browser-based botnets were discovered. I also started to
work with the Mastermind Puzzle in 1996.

\cite{jj-ppsn96}
setting the state of the art several times until today. Our latest hints at 
literary engineering.

\cite{2014arXiv1403.3084G}

Lately, I'm getting interested in mining GitHub repositories looking
at the health of communities and their evolution and other
issues. This is the first paper I published on the topic
\cite{2016arXiv160107862M}
but, since that one, we have produced new results on the complex
nature of repositories using the same repository-mining techniques.

These regular contributions to journals and open publications have
netted nearly 6000 citations and and $H=31$ according to Google Scholar (link provided on
the sidebar).
%\printbibliography

\end{document}
